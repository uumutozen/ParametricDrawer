\documentclass[10pt,tikzpicture,notheorems]{beamer}

\usepackage{graphicx}
\usepackage{amsfonts}
\usepackage{amsthm}
\usepackage[backend=bibtex,maxbibnames=9,maxcitenames=4]{biblatex} %giveninits=true,
\usepackage{etoolbox}
\usepackage{filecontents}
\usepackage{mathtools}
\usepackage{ragged2e}
\justifying

\usepackage{changepage}
\usepackage{xcolor}
\definecolor{pythongreen}{RGB}{106,160,93}
\definecolor{pythonorange}{RGB}{217,118,27}
\definecolor{pythongray}{RGB}{140,139,135}
\usepackage{listings}
\definecolor{codegreen}{rgb}{0.43,0.57,0.49}
\definecolor{codegray}{rgb}{0.5,0.5,0.5}
\definecolor{codepurple}{rgb}{0.4,0.43,0.58}
\definecolor{codeorange}{rgb}{0.70,0.57,0.51}
%\definecolor{backcolour}{rgb}{0.95,0.95,0.92}
\definecolor{backcolour}{rgb}{0.11,0.12,0.13}
\definecolor{plaincode}{rgb}{0.69,0.72,0.75}

\lstdefinestyle{mystyle}{
    backgroundcolor=\color{backcolour},
    commentstyle=\color{codegreen},
    keywordstyle=\color{codeorange},
    numberstyle=\tiny\color{codegray},
    stringstyle=\color{codepurple},
    basicstyle=\ttfamily\footnotesize\color{plaincode},
    breakatwhitespace=false,
    breaklines=true,
    captionpos=b,
    keepspaces=true,
    numbers=left,
    numbersep=5pt,
    showspaces=false,
    showstringspaces=false,
    showtabs=false,
    tabsize=2
}

\lstset{style=mystyle}


\setbeamerfont{quote}{shape=\upshape,family=\ttfamily}


%\inserttheoremnumber


\usetheme{Warsaw}
\usecolortheme{whale}

\let\Tiny=\tiny

\hypersetup{bookmarksdepth=4,bookmarksnumbered=true,bookmarksopen=true}

\beamertemplatenavigationsymbolsempty

\setbeamersize{text margin left=5mm,text margin right=5mm}

\setbeamertemplate{qed symbol}{\rule{0.2cm}{0.2cm}}
\setbeamertemplate{theorems}[numbered]
\setbeamertemplate{caption}[numbered]
%\setbeamertemplate{enumerate item}{(\roman{enumi})}

\setbeamertemplate{background canvas}[vertical shading][bottom=white!10,top=blue!10]
\usefonttheme[onlymath]{serif}

\makeatletter
\renewcommand\eqref[1]{%
  \textup{\usebeamercolor[fg]{structure}\tagform@{\ref{#1}}}%
}

%\DefineBibliographyStrings{english}{%
%  references = {Kaynak\c{c}a},
%  and = {ve},
%  andothers = {vd.}
%}

\DefineBibliographyExtras{english}{
  \let\finalandcomma=\empty
}

%\undef{\theorem}
\newtheorem{theorem}{Theorem}
\newtheorem{corollary}{Corollary}

%\undef{\lemma}
\newtheorem{lemma}{Lemma}

%\undef{\example}
\theoremstyle{example}
\newtheorem{example}{Example}

%\undef{\definition}
\theoremstyle{definition}
\newtheorem{definition}{Definition}
\newtheorem{property}{Property}

%

%http://tex.stackexchange.com/a/123992/9955
\makeatletter
\def\th@remark{%
    \normalfont % body font
    \setbeamercolor{block title alert}{use=alerted text,fg=white,bg=alerted text.fg!75!black}
    \setbeamercolor{block body alert}{parent=normal text,use=block title alerted,bg=block title alerted.bg!10!bg}
    \def\inserttheoremblockenv{alertblock}
  }
\makeatother

\theoremstyle{remark}
\newtheorem{remark}{Remark}

%\setbeamertemplate{bibliography item}[text]

\setbeamertemplate{bibliography item}{%
  \ifboolexpr{test {\ifentrytype{book}} or test {\ifentrytype{thesis}}}
    {\setbeamertemplate{bibliography item}[book]}
    {\ifentrytype{online}
       {\setbeamertemplate{bibliography item}[online]}
       {\setbeamertemplate{bibliography item}[article]}}%
  \usebeamertemplate{bibliography item}
}

\defbibenvironment{bibliography}
  {\list{}
     {\settowidth{\labelwidth}{\usebeamertemplate{bibliography item}}%
      \setlength{\leftmargin}{\labelwidth}%
      \setlength{\labelsep}{\biblabelsep}%
      \addtolength{\leftmargin}{\labelsep}%
      \setlength{\itemsep}{\bibitemsep}%
      \setlength{\parsep}{\bibparsep}}}%
  {\endlist}
  {\item}

%http://tex.stackexchange.com/questions/28760/custom-beamer-blocks
\newenvironment<>{refblock}[1]{%
  \begin{actionenv}#2%
      \def\insertblocktitle{#1}%
      \par%
      \mode<presentation>{%
       \setbeamercolor{block title}{fg=white,bg=orange!20!black}
       \setbeamercolor{block body}{fg=black,bg=yellow!20}
     }%
      \vfill
      \usebeamertemplate{block begin}}
    {\par\usebeamertemplate{block end}\end{actionenv}}

\newenvironment<>{tableblock}[1]{%
  \begin{actionenv}#2%
      \def\insertblocktitle{#1}%
      \par%
      \mode<presentation>{%
    \setbeamercolor{block title}{fg=white,bg=orange}
    \setbeamercolor{block body}{fg=black,bg=orange!20}
     }%
      \vfill
      \usebeamertemplate{block begin}}
    {\par\usebeamertemplate{block end}\end{actionenv}}

\renewcommand*{\newunitpunct}{\addcomma\space}

\DeclareFieldFormat[article]{title}{#1}
\DeclareFieldFormat[article]{citetitle}{#1}
\DeclareFieldFormat[article]{volume}{\mkbibbold{#1}} %\bibstring{volume}\,{#1}
\DeclareFieldFormat[article]{number}{\bibstring{number}\thinspace{#1}}
\DeclareFieldFormat[article]{year}{\mkbibparens{#1}}
\DeclareFieldFormat[article]{pages}{#1}

\DeclareFieldFormat[book]{title}{#1} %\mkbibemph{#1}
\DeclareFieldFormat[book]{citetitle}{#1} %\mkbibemph{#1}
\DeclareFieldFormat[book]{pages}{#1}
\DeclareFieldFormat[book]{volume}{\bibstring{volume}\thinspace{#1}}

\DeclareFieldFormat[incollection]{title}{#1}
\DeclareFieldFormat[incollection]{citetitle}{#1}
\DeclareFieldFormat[incollection]{booktitle}{#1} %\mkbibemph{#1}
\DeclareFieldFormat[incollection]{pages}{#1}

\DeclareFieldFormat[inproceedings]{title}{#1}
\DeclareFieldFormat[inproceedings]{citetitle}{#1}
\DeclareFieldFormat[inproceedings]{booktitle}{#1} %\mkbibemph{#1}
\DeclareFieldFormat[inproceedings]{pages}{#1}

\DeclareFieldFormat[thesis]{title}{#1} %\mkbibemph{#1}
\DeclareFieldFormat[thesis]{citetitle}{#1} %\mkbibemph{#1}
\DeclareFieldFormat[thesis]{pages}{#1}
\DeclareFieldFormat[thesis]{annote}{#1}
\DeclareFieldFormat[online]{title}{#1}
\DeclareFieldFormat[online]{citetitle}{#1}
\DeclareFieldFormat[online]{url}{\url{#1}} %\mkbibacro{URL}\addcolon\space\url{#1}


\renewbibmacro{in:}{}


\renewbibmacro*{journal+issuetitle}{%
  \printfield{journaltitle}%
  \setunit*{\space}%
  \printfield{volume}%
  \setunit*{\space}%
  \printfield{year}% Added
  \setunit*{\addcomma\space}%
  \printfield{number}%
  \newunit
}

\renewbibmacro*{publisher+location+date}{%
    \printlist{publisher}%
    \setunit*{\addcomma\space}%
    \printlist{location}%
    \setunit*{\addcomma\space}%
    \printdate%
}

\newenvironment{referencelist}{%
 \setbeamerfont{description item}{size=\tiny}
 \setbeamersize{description width=0cm}
 \begin{description}\tiny}{\end{description}
}

\newenvironment{articleitem}[1]{%
 {\usebeamercolor[fg]{bibliography entry author}\usebeamerfont{bibliography entry author}\cite{#1}}
 {\usebeamercolor[fg]{bibliography entry author}\usebeamerfont{bibliography entry author}\citeauthor{#1},}
 {\usebeamercolor[fg]{bibliography entry title}\usebeamerfont{bibliography entry title}\citetitle{#1},}
 {\usebeamercolor[fg]{bibliography entry location}\usebeamerfont{bibliography entry location}\citefield{#1}[article]{journaltitle}, \citefield{#1}[article]{volume} \citeyear{#1}, \citefield{#1}[article]{number}, \citefield{#1}{pages}.}
}


\newenvironment{bookitem}[1]{%
 {\usebeamercolor[fg]{bibliography entry author}\usebeamerfont{bibliography entry author}\cite{#1}}
 {\usebeamercolor[fg]{bibliography entry author}\usebeamerfont{bibliography entry author}\citeauthor{#1},}
 {\usebeamercolor[fg]{bibliography entry title}\usebeamerfont{bibliography entry title}\citetitle{#1},}
 {\usebeamercolor[fg]{bibliography entry location}\usebeamerfont{bibliography entry location}\citelist{#1}{publisher}, \citeyear{#1}.}
}


\newenvironment{thesisitem}[1]{%
 {\usebeamercolor[fg]{bibliography entry author}\usebeamerfont{bibliography entry author}\cite{#1}}
 {\usebeamercolor[fg]{bibliography entry author}\usebeamerfont{bibliography entry author}\citeauthor{#1},}
 {\usebeamercolor[fg]{bibliography entry title}\usebeamerfont{bibliography entry title}\citetitle{#1},}
 {\usebeamercolor[fg]{bibliography entry location}\usebeamerfont{bibliography entry location}\citefield{#1}[thesis]{type}, \citelist{#1}{institution}, \citeyear{#1}.}
}

\newenvironment{webitem}[1]{%
 {\usebeamercolor[fg]{bibliography entry author}\usebeamerfont{bibliography entry author}\cite{#1}}
 {\usebeamercolor[fg]{bibliography entry author}\usebeamerfont{bibliography entry author}\citeauthor{#1},}
 {\usebeamercolor[fg]{bibliography entry title}\usebeamerfont{bibliography entry title}\citetitle{#1},}
 {\usebeamercolor[fg]{bibliography entry location}\usebeamerfont{bibliography entry location}\citeurl{#1}.}
}

%\addbibresource{\jobname.bib}

\DeclareMathOperator{\Log}{Log}

\newcommand{\dd}{\mathrm{d}}
\newcommand{\dD}{\mathrm{D}}
\newcommand{\ef}{\mathrm{e}}
\newcommand{\imn}{\mathrm{i}}
\newcommand{\Lop}{\mathrm{L}}
\newcommand{\Iop}{\mathrm{I}}

\newcommand{\cnt}[1]{\mathrm{C}^{#1}}

\newcommand{\N}{\mathbb{N}}
\newcommand{\R}{\mathbb{R}}
\newcommand{\Z}{\mathbb{Z}}
\newcommand{\C}{\mathbb{C}}

\renewcommand{\Re}{\operatorname{Re}}
\renewcommand{\Im}{\operatorname{Im}}

\renewcommand*{\thefootnote}{\fnsymbol{footnote}}

\title[{\makebox[.45\paperwidth]{GCode Animation with \texttt{Python} \hfill%
\insertframenumber/\inserttotalframenumber}}]{GCode Animation with \texttt{Python}:\\ G0-G1, G2-G3, G05\footnote{Supervisor: Dr.~Ba\c{s}ak Karpuz}}

\titlegraphic{\includegraphics[width=0.15\textwidth]{deulogo.eps}}
%\titlegraphic{\includegraphics[width=0.20\textwidth,natwidth=433,natheight=405]{deulogo.png}}

\author[U.~\"{O}zen]{Umut \"{O}zen}

\institute{Department of Mathematics,\\
Dokuz Eyl\"{u}l University.
%\.{I}zmir, Turkey.
} %35160

\date{2024-2025 - Fall Semester}

\begin{document}

\begin{frame}{}{}

\titlepage

\end{frame}

%

\begin{frame}{}{}
\frametitle{\abstractname}

\begin{block}{\abstractname}
In this presentation,
we introduce some GCodes for CNC (Computer Numerical Control) machines.
%G0, G1, G2, G3, G5
%https://marlinfw.org/docs/gcode/G000-G001.html
Then, we introduce mathematical background related to motion and behavior GCodes of the CNC machine.
Finally, we give our \texttt{Python} codes that creates an animation for a tangential cutter by reading a \texttt{.gcode} file.
\end{block}

\end{frame}

%

\begin{frame}{}{}
\frametitle{Outline of the Talk}

Below, we explain the procedure on
how we retrieve information from weekly course schedules in \texttt{DEBIS}.
\begin{enumerate}
\item Examining the target web page on web browser
    \begin{enumerate}[i]
    \item Playing with the objects
    \item Inspecting the source
    \end{enumerate}
\item Reading data from the internet by using \texttt{Python}
    \begin{enumerate}[i]
    \item Getting list of academics
    \item Reading department information
    \item Getting weekly course data
    \end{enumerate}
\item Parsing data from the source by using \texttt{Python}
    \begin{enumerate}[i]
    \item Extracting timetable entries
    \end{enumerate}
\item Saving data to an \texttt{Excel Sheet} by using \texttt{Python}
\item Processing data in the excel sheet by using \texttt{Python}
\end{enumerate}

\end{frame}

%

\section{GCodes}

\begin{frame}{}{}
\frametitle{\insertsectionhead}
\framesubtitle{\insertsubsectionhead}

%GCode general information
%Some CNC/Vinyl Cutter photos
GCode is a programming language used to control the movements and operations of CNC (Computer Numerical Control) machines.
It specifies how the machine should move, how fast it should move, and what operations it should perform.


\end{frame}

%%

\subsection{Some GCodes: Linear Motion G0-G1}

\begin{frame}{}{}
\frametitle{\insertsectionhead}
\framesubtitle{\insertsubsectionhead}

\textellipsis
https://marlinfw.org/docs/gcode/G000-G001.html

\end{frame}

%%

\subsection{Some GCodes: Circular Motion G2-G3}

\begin{frame}{}{}
\frametitle{\insertsectionhead}
\framesubtitle{\insertsubsectionhead}

\textellipsis
https://marlinfw.org/docs/gcode/G002-G003.html

\end{frame}

%%

\subsection{Some GCodes: Cubic Spline G5}

\begin{frame}{}{}
\frametitle{\insertsectionhead}
\framesubtitle{\insertsubsectionhead}

\textellipsis
https://marlinfw.org/docs/gcode/G005.html

\end{frame}

%%

\section{Curves Related to the GCodes: G0-G1, G2-G3, G5}

%%

\subsection{Some GCodes: Linear Motion G0-G1}

\begin{frame}{}{}
\frametitle{\insertsectionhead}
\framesubtitle{\insertsubsectionhead}

The line segment starting from the point $P_{0}$ and ending at the point $P_{1}$
\begin{equation}
\alpha(t):=P_{0}+t(P_{1}-P_{0}),\quad{}0\leq{}t\leq1,\nonumber
\end{equation}
whose tangent slope is the constant value
\begin{equation}
\alpha^{\prime}(t):=(P_{1}-P_{0}),\quad{}0\leq{}t\leq1,\nonumber
\end{equation}

\end{frame}

%%

\subsection{Some GCodes: Circular Motion G2-G3}

\begin{frame}{}{}
\frametitle{\insertsectionhead}
\framesubtitle{\insertsubsectionhead}

The circular arc centered at the point $P_{0}$ with radius $r$,
starting at an angle $\theta_{0}$ and ending at an angle $\theta_{1}$
\begin{equation}
\alpha(t):=P_{0}+r\bigl(\cos(t),\sin(t)\bigr),\quad\theta_{0}\leq{}t\leq\theta_{1},\nonumber
\end{equation}
whose tangent slope is the circular curve
\begin{equation}
\alpha^{\prime}(t):=r\bigl(-\sin(t),\cos(t)\bigr),\quad\theta_{0}\leq{}t\leq\theta_{1}.\nonumber
\end{equation}

\end{frame}

%%

\subsection{Some GCodes: Cubic Spline G5}

\begin{frame}{}{}
\frametitle{\insertsectionhead}
\framesubtitle{\insertsubsectionhead}

Cubic B\'{e}zier curve starting at the point $P_{0}$, ending at the point $P_{3}$
and with the additional control points $P_{1}$ and $P_{2}$
\begin{equation}
\alpha(t):=\sum_{i=0}^{3}\binom{3}{i}(1-t)^{3-i}t^{i}P_{i},\quad{}0\leq{}t\leq1,\nonumber
\end{equation}
whose tangent curve is
\begin{equation}
\begin{aligned}[]
\alpha^{\prime}(t):={}&\sum_{i=0}^{2}\binom{3}{i}(3-i)(1-t)^{2-i}t^{i}P_{i}\\
{}&+\sum_{i=1}^{3}\binom{3}{i}i(1-t)^{3-i}t^{i-1}P_{i},\quad{}0\leq{}t\leq1.
\end{aligned}\nonumber
\end{equation}
\end{frame}

%%

\section{Animation Program Code}

\begin{frame}{Animation Program Code}
    \lstinputlisting[language=Python, firstline=1, lastline=17]{gcode-scatter(2025.01.06).py}
\end{frame}

%%

\begin{frame}{Animation Program Code}
    \lstinputlisting[language=Python, firstline=20, lastline=24]{gcode-scatter(2025.01.06).py}
    \lstinputlisting[language=Python, firstline=27, lastline=29]{gcode-scatter(2025.01.06).py}
    \lstinputlisting[language=Python, firstline=27, lastline=29]{gcode-scatter(2025.01.06).py}
\end{frame}

%%

%\begin{frame}{Animation Program Code}
%    \lstinputlisting[language=Python, firstline=1, lastline=17]{gcode-scatter(2025.01.06).py}
%\end{frame}

%%

\begin{frame}
%
Thank you very much for your interest to our talk.\bigskip\\
%We also would like to express our sincere thanks to\smallskip\\
%
%\begin{quote}
%\begin{adjustwidth}{3em}{1em}
%\colorbox{black}{\textcolor{white}{{\Large print(}\textcolor{pythongreen}{\Large "Dr.~Volkan \"{O}\u{g}er"}\textcolor{white}{\Large )}}}
%\end{adjustwidth}
%\end{quote}
%for giving us some directions to make this project complete.
%
\end{frame}

\end{document}
