\chapter{PYTHON}

In this chapter we will be talking about the Pyhton libraries we used in the code (see \cite{pythonweb}).

\section*{EZDXF}

EZDXF is a Python interface to the DXF (drawing interchange file) format developed by Autodesk, ezdxf allows developers to read and modify existing DXF documents or create new DXF documents.

The main objective in the development of EZDXF was to hide complex DXF details from the programmer but still support most capabilities of the DXF format. Nevertheless, a basic understanding of the DXF format is required, also to understand which tasks and goals are possible to accomplish by using the DXF format.

We used EZDXF library to read data from an existing DXF document.

\begin{lstlisting}[language=Python]
import ezdxf
doc=ezdxf.readfile(filename)
\end{lstlisting}

In this code block we access the model space of the DXF file, where all the entities (shapes) are stored.

\begin{lstlisting}[language=Python]
msp=doc.modelspace()
\end{lstlisting}

That way we can read our entities types and with the read types we can carry out operations.

For example:
\begin{lstlisting}[language=Python]
for entity in msp:
if entity.dxftype()=="LINE":
    print_entity(entity)
\end{lstlisting}


\section*{Matplotlib}
 Matplotlib makes it easier to analyze metrics when they are presented in plots,
 which clarify dynamics and trends in seconds, rather than in a huge set of data.
 Matplotlib is a tool that allows your program to speak in a visual way and turn your data into graphs,
 pie charts, bar charts, and more.
 We used this library to visualize the data's inside the DXF files that we read by using the EZDXF library.
 For example:
\begin{lstlisting}[language=Python]
plt.plot(x_bspline_points,y_bspline_points,`b-')
plt.plot(x_bspline_points,y_bspline_points,`bo-',label=`BSpline Points',alpha=0.75)
\end{lstlisting}        
With this example we draw B-Spline points on $xy$ coordinate plane according to their $x$ and $y$ values.
 
\section*{Numpy}

NumPy is a fundamental library in Python, renowned for its support for large,
multi-dimensional arrays and matrices,
along with a comprehensive collection of mathematical functions to operate on these arrays.

This function is able to create an array:
\begin{lstlisting}[language=Python]
create_Array=numpy.array(object,dtype=None,*,copy=True,order=`K',subok=False,ndmin=0,like=None)
\end{lstlisting}   

This function is able to return one of eight different matrix norms, or one of an infinite number of vector norms (described below), depending on the value of the ord parameter:
\begin{lstlisting}[language=Python]
create_matrixnorm=linalg.norm(x,ord=None,axis=None,keepdims=False)
\end{lstlisting}

This function is able to find the trigonometric inverse cosine, this means that if $y=\cos(x)$ then $x=\arccos(y)$:
\begin{lstlisting}[language=Python]
x_arcos=numpy.arccos(x,/,out=None,*,where=True,casting=`same_kind',order=`K',dtype=None,subok=True[,signature,extobj])=<ufunc`arccos'>
\end{lstlisting}   

This function is able to dot product of two arrays:
\begin{lstlisting}[language=Python]
xy_dotproduct=numpy.dot(x,y,out=None)
\end{lstlisting}   

This function is able to creates an array of evenly spaced numbers over a specified interval:
\begin{lstlisting}[language=Python]
t=numpy.linspace(start,stop,num=50,endpoint=True,retstep=False,dtype=None,axis=0)
\end{lstlisting}  

\section*{Math}

The math library in Python provides a suite of mathematical functions derived from the C standard library,
catering to operations like trigonometric calculations, logarithms, and factorial computations.
It enables precise floating-point arithmetic
and complex mathematical tasks, such as calculating square roots, sines, and exponential functions.

We use just math.degree function in this library:
\begin{lstlisting}[language=Python]
x_degrees=math.degrees(x)
\end{lstlisting}  

This function convert angles from radians to degrees

\section*{SciPy}

SciPy is an advanced Python library that builds on NumPy's capabilities,
offering a vast array of mathematical algorithms and functions for science and engineering.
It includes tools for optimization, linear algebra, integration, and more,
making it essential for tasks requiring sophisticated numerical computations.

In our project we used Bspline module in SciPy for spline interpolation and approximation.
This module helps us to create and manipulate B-Spline curves.
Also we perform some operations differentiation, and integration of B-Splines with this module.

This function is able to create B-Spline:
\begin{lstlisting}[language=Python]
scipy.interpolate.BSpline(t,c,k,extrapolate=True,axis=0)
\end{lstlisting}

This function returns a B-Spline representing the derivative:
\begin{lstlisting}[language=Python]
BSpline.derivative(nu=1)
\end{lstlisting}

Now, we are presenting our Python codes for the Tangential Cutter.

\lstinputlisting[language=Python,caption=Ellipse Example]{ellipse+smooth.py}
\bigskip
\lstinputlisting[language=Python,caption=B-Spline Example]{splines-only.py}
