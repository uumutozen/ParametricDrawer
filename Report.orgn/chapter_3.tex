\chapter{MATHEMATICS}

In this chapter, we explain the figures' mathematical characteristics that we created with the fusion applications DXF files.
\section{Planar Curves}
A planar curve is a curve that, although it may exist in a higher-dimensional space, can be projected or embedded onto a plane without intersections or overlapping parts, effectively making it topologically equivalent to a curve that lies completely within a two-dimensional plane. This property means that even if the curve's original formulation is in three dimensions (or higher), its structure allows it to be represented in two dimensions without loss of essential characteristics, such as connectivity or general shape.
\section{Tangent Lines}
In the realm of mathematical analysis, the tangent line to a curve at a particular point is a critical construct that bridges geometric intuition with analytical rigor. This line touches the curve at a single point and matches the slope of the curve precisely at that point. This connection between a curve and its tangent line can be elucidated using the principles of calculus, specifically through the derivative of the function that describes the curve.

\noindent\textbf{Derivative and the Slope of the Tangent Line.}
The foundation of understanding tangent lines lies in the concept of the derivative. The derivative of a function at any point gives the slope of the tangent line to the function at that particular point. Mathematically, if a function
$y=f(x)$ represents a curve, the derivative
$f^{\prime}(x)$ denotes the slope of the tangent line at any given value of $x$.

\noindent\textbf{Formulating the Equation of a Tangent Line.}
To derive the equation of a tangent line to a curve at a specific point
$x=a$, we employ the following systematic methodology:
\begin{itemize}
\item \textbf{Determine the Derivative.}
    Compute $f^{\prime}(a)$, the derivative of $f$ evaluated at the point $x=a$.
    This value represents the slope of the tangent line at $x=a$.
\item \textbf{Utilize the Point-Slope Form.}
    The point-slope formula, $y-y_{0}=m(x-x_{0})$,
    where $m$ is the slope and $(x_{0},y_{0})$ is a known point on the line, is essential here. For the tangent line,
    we substitute $m=f^{\prime}(a)$ and $(x_{0},y_{0})=(a,f(a))$
\item \textbf{Substitute and Rearrange.}
    Incorporating these values into the point-slope equation provides
    \begin{equation}
    y=f^{\prime}(a)(x-a)+f(a),\nonumber
    \end{equation}
    which upon rearrangement, yields the equation of the tangent line at $x=a$.
\end{itemize}


\section{Some Fundamental Curves}

\subsection{Lines}
A line is a one-dimensional figure, which has length but no width.
Line is made of a set of points which is extended in opposite directions infinitely.
It is determined by two points in a two-dimensional plane.
The two points which lie on the same line are said to be collinear points.

The general equation of straight line is
\begin{equation}
ax+by+c=0\nonumber
\end{equation}
where $a,b,c$ are constants, $x$ and  $y$ are variables and $(-\frac{a}{b})$ is slope.
This equation can also be represented in the parametric form as
\begin{equation}
(t,-\tfrac{a}{b}t-\tfrac{c}{b})\quad\text{for}\ t\in\mathbb{R}.\nonumber
\end{equation}


\subsection{Circles and Circular Arcs}
In geometry, a circle is a special kind of ellipse which the set of all the points in the plane is equidistant from a given point called {\textquotedblleft}centre{\textquotedblright}.
Every line that passes through the circle forms the line of reflection symmetry.
Also, it has rotational symmetry around the centre for every angle.
The circle formula is
\begin{equation}
(x-a)^{2}+(y-b)^{2}=r^{2}\nonumber
\end{equation}
where $(x,y)$ are the coordinate points,
$(a,b)$ is the coordinate of the centre of a circle,
$r$ is the radius of a circle.
The equation of a circle can also be represented in the parametric form as
\begin{equation}
\bigl(r\cos(t)+a,r\sin(t)+b\bigr)\quad\text{for}\ 0\leq{}t<2\pi.\nonumber
\end{equation}
Let $(p,q)$ be a point on a the circle,
then the tangent line of the circle through the points $(p,q)$ is given by
\begin{equation}
\bigl(r\cos(t)+a,r\sin(t)+b\bigr)\quad\text{for}\ 0\leq{}t<2\pi.\nonumber
\end{equation}
where $(p,q)=\bigl(r\cos(\theta_{0})+a,r\sin(\theta_{0})+b\bigr)$ for some $0\leq\theta_{0}<2\pi$.

\subsection{Circular Arcs}
Circular arcs are portions of the circumference of a circle.
In addition to the data for circles, they are defined by two angles:
the starting angle and the ending angle.
Therefore, their parametric formula can be given by
\begin{equation}
\bigl(r\cos(t)+a,r\sin(t)+b\bigr)\quad\text{for}\ \theta_{1}\leq{}t\leq\theta_{2}.\nonumber
\end{equation}


\subsection{Ellipses}
An ellipse is a set of points $(x, y)$ in a Cartesian plane satisfying an equation of the form \begin{equation}
\biggl(\frac{x}{a}\biggr)^{2}+\biggl(\frac{y}{b}\biggr)^{2}=1,\nonumber
\end{equation}
where $a$ is the distance from the origin to the end of
the $x$-axis on the ellipse (semimajor axis),
$b$ is the distance from the origin to the end of
the $y$-axis on the ellipse (semiminor axis), and $a>b$.
The equation of an ellipse can have other forms, but this one,
with the center at the origin and the major axis coinciding with one of the coordinate axes,
is the simplest.
The equation of an ellipse can also be represented in the parametric form as
\begin{equation}
\bigl(a\cos(t),b\sin(t)\bigr)\quad\text{for}\ 0\leq{}t<2\pi.\nonumber
\end{equation}


\subsection{Spline}
Splines are a mathematical and computational tool used extensively for creating smooth and flexible curves through a given set of points, or for function approximation across a domain. In its most basic form, a spline is a piecewise-defined function, typically polynomial in each piece. The overall curve aims to achieve smoothness and minimal curvature, making splines especially valuable in computer graphics, data fitting, and numerical simulation domains.

\subsubsection{B-Spline}

B-Splines, or Basis Splines, offer a robust way to construct these piecewise polynomials,
providing great flexibility and precision.
A B-spline is defined by its order, degree, control points, and a knot vector \cite{allaire2008numerical}.
\begin{itemize}
\item \textbf{Degree (p).} This is the degree of the polynomial in each segment.
    The degree plus one equals the order of the spline.
    For instance, a cubic spline has a degree of three.
\item \textbf{Control Points.} These are the points in a multidimensional space that guide the shape of the spline.
    The spline does not necessarily pass through these points (except in certain conditions like for Bezier curves),
    but they influence the curvature and bending of the spline.
\item \textbf{Knot Vector.} This is a sequence of parameter values that determines where
    and how the control points affect the B-spline curve.
    The knot vector divides the parametric space into intervals
    and influences the continuity and smoothness across the segments of the spline.
\end{itemize}

A B-spline curve is mathematically expressed as:
\begin{equation}
f(t):=\sum_{i=0}^{n}P_{i}B_{i,p}(t),\nonumber
\end{equation}
where
\begin{itemize}
\item $f(t)$ is the point on the B-spline curve at a parameter value $t$.
\item $P_{i}$ are the control points.
\item $B_{i,p}$ are the B-spline basis functions of degree $p$ related to the knot vector.
\end{itemize}
The B-spline basis functions, $B_{i,p}(t)$, are recursively defined by
\begin{equation}
B_{i,0}(t):=
\begin{cases}
1 & \text{if}\ t_{i}\leq{}t < t_{i+1} \\
0 & \text{otherwise}
\end{cases}\nonumber
\end{equation}
and
\begin{equation}
B_{i,p}(t):=\frac{t-t_{i}}{t_{i+p}-t_{i}}B_{i,p-1}(t)+\frac{t_{i+p+1}-t}{t_{i+p+1}-t_{i+1}}B_{i+1,p-1}(t)\quad\text{for}\ n\in\mathbb{N}.\nonumber
\end{equation}
The elegant construction of B-splines using these basis functions allows for local control of the curve
(adjusting one control point only affects the curve locally, within a few segments),
and varying the knot vector can produce different levels of smoothness and curve complexities.
This makes B-splines incredibly powerful and flexible for practical applications like CAD/CAM systems,
animation, and shape modeling. 