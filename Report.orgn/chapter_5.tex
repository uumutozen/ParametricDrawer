\chapter{G-CODES: The Language of CNC Machines}


G-Code, short for {\textquotedblleft}Geometric Code{\textquotedblright} or {\textquotedblleft}Gestalt Code{\textquotedblright}, is a language used to control CNC (Computer Numerical Control) machines such as 3D printers, CNC mills, lathes, routers, and laser cutters.
The idea originated in the mid-20th century during the development of the first numerical control (NC) systems for machining.
While the specific individuals responsible for their invention may not be easily identifiable, G-Codes emerged from collaborative efforts among engineers, researchers, and programmers, particularly at institutions like the Massachusetts Institute of Technology (MIT) and within the manufacturing industry during the 1950s and 1960s  (see \cite{marlinfwweb}).
G-Code consists of a series of commands or instructions that tell the machine how to move, position, and operate the cutting tool or print head to create a desired object or part.

Here are some key aspects of G-Codes:

\noindent\textbf{Commands.}
G-Code commands are alphanumeric codes preceded by a letter, such as G, M, T, or S, followed by numerical values or parameters.
Each command corresponds to a specific action or function, such as movement, tool selection, spindle speed, or coolant control.

\noindent\textbf{Coordinate System.}
G-Code uses a coordinate system to specify the position and movement of the machine's axes.
Commonly, this system consists of three linear axes $(x,y,z)$ for 3D printers and CNC mills,
along with additional axes for rotary or angular movement in certain machines.

\noindent\textbf{Motion Control.}
G-Code commands control various aspects of motion, including linear moves, arc moves,
rapid positioning, feed rates (speed of movement), acceleration, and deceleration.
These commands dictate how the machine moves the cutting tool or print head to create the desired shapes and contours.

\noindent\textbf{Tool Control.}
G-Code includes commands for controlling the cutting tool or print head,
such as tool selection, tool change, spindle speed (RPM), direction of rotation, and tool engagement.
These commands determine how the machine operates and interacts with the material being processed.

\noindent\textbf{Program Structure.}
G-Code programs are typically structured as a series of sequential commands organized into blocks or lines of code.
Each line of code represents a specific action or operation to be performed by the machine.
G-Code programs can also include conditional statements, loops, and subroutines for more complex operations.

\noindent\textbf{Post-Processing.}
G-Code files are generated by CAM (Computer-Aided Manufacturing) software based on a 3D model or design.
The CAM software translates the design into a series of G-Code commands tailored to the specific machine and manufacturing process.
These G-Code files can then be loaded into the CNC machine's controller for execution.

Overall, G-Code is a fundamental component of CNC machining and 3D printing,
providing precise control over the manufacturing process
and enabling the creation of complex and intricate parts with high accuracy and repeatability.

\section{Some G-Codes}
In this section, we will review some G-Codes that were used in our project.

\noindent\textbf{G0.}
The G0 command in CNC programming instructs the machine to rapidly move to a specified position
without machining along the way, operating at its maximum traverse speed.
Used primarily for repositioning the tool or machine swiftly between machining locations
or to a starting point for a new operation, G0 commands are essential for efficient CNC machining processes.
However, it's crucial to consider safety, as rapid movements can lead to collisions
or hazards if not carefully controlled within the machining program.

\noindent\textbf{G1.}
The G1 command in CNC programming directs the machine to move in a straight line from its current position to a specified endpoint at a controlled feed rate, allowing for material removal during the motion. This command is fundamental for shaping and machining operations, as it enables precise control over the tool's path and speed, resulting in accurate workpiece fabrication. Unlike the rapid movements of G0, G1 commands involve cutting or material removal, making them essential for achieving the desired dimensions and surface finish of the machined part. As with all CNC commands, proper programming and consideration of safety measures are crucial to ensure efficient and safe machining processes.

\noindent\textbf{G92.}
The G92 command in CNC programming allows for the establishment of a new reference position or coordinate system on the machine. When issued, G92 instructs the machine to set its current position to the specified coordinates, effectively resetting the coordinate system origin. This command is particularly useful for workpiece setup and zero-point definition, enabling operators to designate a specific location as the reference point for subsequent machining operations. By redefining the coordinate system origin, G92 facilitates accurate positioning and machining of parts, streamlining production processes and enhancing overall precision. Proper utilization of the G92 command is essential for ensuring consistency and repeatability in CNC machining tasks, ultimately contributing to the production of high-quality components.

\noindent\textbf{M92.}
The M92 code is typically used in 3D printing rather than CNC machining. In 3D printing, M92 is a command that sets the steps per unit for each axis of the printer. This command allows calibrating the printer to ensure accurate movement according to the specified dimensions in the design files. By adjusting the steps per unit, users can correct any discrepancies between the physical movement of the printer's motors and the intended movements in the 3D model. Proper calibration with the M92 command is crucial for achieving precise and accurate prints, ensuring that the final objects match the intended dimensions and details specified in the design.


%E88.8889; Calibrate E Stepper Motor, E445; Calibrate E Stepper Motor Back

\noindent\textbf{M302.}
The M302 code is used in some 3D printing firmware to enable or disable cold extrusion prevention. In 3D printing, cold extrusion prevention is a safety feature designed to prevent the extruder from operating if the hotend temperature is below a certain threshold. The M302 command allows users to override this safety measure, enabling extrusion even when the hotend is cold. This feature can be useful for troubleshooting or performing maintenance tasks that require manual control over the extrusion process. However, it's important to exercise caution when using M302, as extruding filament at low temperatures can lead to poor print quality, clogs, or other issues. Proper understanding and careful use of the M302 command can help ensure safe and efficient 3D printing operations.

%P1; Disable Cold Extrusion Prevention, P0; Enable Cold Extrusion Prevention

